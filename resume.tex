\documentclass[10pt, letterpaper]{article}

% Packages:
\usepackage[
    ignoreheadfoot, % set margins without considering header and footer
    top=2 cm, % seperation between body and page edge from the top
    bottom=2 cm, % seperation between body and page edge from the bottom
    left=2 cm, % seperation between body and page edge from the left
    right=2 cm, % seperation between body and page edge from the right
    footskip=1.0 cm, % seperation between body and footer
    % showframe % for debugging 
]{geometry} % for adjusting page geometry
\usepackage{titlesec} % for customizing section titles
\usepackage{tabularx} % for making tables with fixed width columns
\usepackage{array} % tabularx requires this
\usepackage[dvipsnames]{xcolor} % for coloring text
\definecolor{primaryColor}{RGB}{0, 0, 0} % define primary color
\usepackage{enumitem} % for customizing lists
\usepackage{fontawesome5} % for using icons
\usepackage{amsmath} % for math
\usepackage[
    pdftitle={Ashutosh Ramola's CV},
    pdfauthor={Ashutosh Ramola},
    pdfcreator={LaTeX with RenderCV},
    colorlinks=true,
    urlcolor=primaryColor
]{hyperref} % for links, metadata and bookmarks
\usepackage[pscoord]{eso-pic} % for floating text on the page
\usepackage{calc} % for calculating lengths
\usepackage{bookmark} % for bookmarks
\usepackage{lastpage} % for getting the total number of pages
\usepackage{changepage} % for one column entries (adjustwidth environment)
\usepackage{paracol} % for two and three column entries
\usepackage{ifthen} % for conditional statements
\usepackage{needspace} % for avoiding page brake right after the section title
\usepackage{iftex} % check if engine is pdflatex, xetex or luatex

% Ensure that generate pdf is machine readable/ATS parsable:
\ifPDFTeX
    \input{glyphtounicode}
    \pdfgentounicode=1
    \usepackage[T1]{fontenc}
    \usepackage[utf8]{inputenc}
    \usepackage{lmodern}
\fi

\usepackage{charter}

% Some settings:
\raggedright
\AtBeginEnvironment{adjustwidth}{\partopsep0pt} % remove space before adjustwidth environment
\pagestyle{empty} % no header or footer
\setcounter{secnumdepth}{0} % no section numbering
\setlength{\parindent}{0pt} % no indentation
\setlength{\topskip}{0pt} % no top skip
\setlength{\columnsep}{0.15cm} % set column seperation
\pagenumbering{gobble} % no page numbering

\titleformat{\section}{\needspace{4\baselineskip}\bfseries\large}{}{0pt}{}[\vspace{1pt}\titlerule]

\titlespacing{\section}{
    % left space:
    -1pt
}{
    % top space:
    0.3 cm
}{
    % bottom space:
    0.2 cm
} % section title spacing

\renewcommand\labelitemi{$\vcenter{\hbox{\small$\bullet$}}$} % custom bullet points
\newenvironment{highlights}{
    \begin{itemize}[
        topsep=0.10 cm,
        parsep=0.10 cm,
        partopsep=0pt,
        itemsep=0pt,
        leftmargin=0 cm + 10pt
    ]
}{
    \end{itemize}
} % new environment for highlights


\newenvironment{highlightsforbulletentries}{
    \begin{itemize}[
        topsep=0.10 cm,
        parsep=0.10 cm,
        partopsep=0pt,
        itemsep=0pt,
        leftmargin=10pt
    ]
}{
    \end{itemize}
} % new environment for highlights for bullet entries

\newenvironment{onecolentry}{
    \begin{adjustwidth}{
        0 cm + 0.00001 cm
    }{
        0 cm + 0.00001 cm
    }
}{
    \end{adjustwidth}
} % new environment for one column entries

\newenvironment{twocolentry}[2][]{
    \onecolentry
    \def\secondColumn{#2}
    \setcolumnwidth{\fill, 4.5 cm}
    \begin{paracol}{2}
}{
    \switchcolumn \raggedleft \secondColumn
    \end{paracol}
    \endonecolentry
} % new environment for two column entries

\newenvironment{threecolentry}[3][]{
    \onecolentry
    \def\thirdColumn{#3}
    \setcolumnwidth{, \fill, 4.5 cm}
    \begin{paracol}{3}
    {\raggedright #2} \switchcolumn
}{
    \switchcolumn \raggedleft \thirdColumn
    \end{paracol}
    \endonecolentry
} % new environment for three column entries

\newenvironment{header}{
    \setlength{\topsep}{0pt}\par\kern\topsep\centering\linespread{1.5}
}{
    \par\kern\topsep
} % new environment for the header

\newcommand{\placelastupdatedtext}{% \placetextbox{<horizontal pos>}{<vertical pos>}{<stuff>}
  \AddToShipoutPictureFG*{% Add <stuff> to current page foreground
    \put(
        \LenToUnit{\paperwidth-2 cm-0 cm+0.05cm},
        \LenToUnit{\paperheight-1.0 cm}
    ){\vtop{{\null}\makebox[0pt][c]{
        \small\color{gray}\textit{Last updated in September 2024}\hspace{\widthof{Last updated in September 2024}}
    }}}%
  }%
}%

% save the original href command in a new command:
\let\hrefWithoutArrow\href

% new command for external links:


\begin{document}
    \newcommand{\AND}{\unskip
        \cleaders\copy\ANDbox\hskip\wd\ANDbox
        \ignorespaces
    }
    \newsavebox\ANDbox
    \sbox\ANDbox{$|$}

    \begin{header}
        \fontsize{25 pt}{25 pt}\selectfont Ashutosh Ramola

        \vspace{5 pt}

        \normalsize
        \mbox{Hyderabad, IN}%
        \kern 5.0 pt%
        \AND%
        \kern 5.0 pt%
        \mbox{\hrefWithoutArrow{mailto:ramola.ashutsoh@gmail.com}{ramola.ashutosh@gmail.com}}%
        \kern 5.0 pt%
        \AND%
        \kern 5.0 pt%
        \mbox{\hrefWithoutArrow{tel:+91-7895545841}{+91-7895545841}}%
        \kern 5.0 pt%
        \AND%
        % \kern 5.0 pt%
        % \mbox{\hrefWithoutArrow{https://yourwebsite.com/}{yourwebsite.com}}%
        % \kern 5.0 pt%
        % \AND%
        \kern 5.0 pt%
        \mbox{\hrefWithoutArrow{https://www.linkedin.com/in/ashutoshramola}{linkedin.com/in/ashutoshramola}}%
        \kern 5.0 pt%
        \AND%
        \kern 5.0 pt%
        \mbox{\hrefWithoutArrow{https://github.com/ashutoshramola}{github.com/ashutoshramola}}%
    \end{header}

    \vspace{5 pt - 0.3 cm}



    \section{Summary}

        \begin{onecolentry}
            Full Stack Aerospace Engineer with expertise in propulsion systems, flight controls, and aircraft development. Specialized in integrating complex aerospace systems including electric propulsion, flight computers, and control architectures. Demonstrated experience in computational fluid dynamics, systems integration, and flight test operations. Proven track record of developing innovative solutions for aerospace applications while ensuring compliance with industry standards and regulations.          
        \end{onecolentry}

        \vspace{0.1 cm}

   \section{Core Competencies}
    \begin{onecolentry}
        \textbf{Aircraft Systems:} Flight Control Systems, Avionics Integration, System Architecture Design, Flight Testing\\
        \textbf{Propulsion:} Electric Propulsion Systems, Turbine Engineering, Thrust Analysis, Thermal Management\\
        \textbf{Software \& Control:} Flight Computer Systems, Real-time Control, Embedded Systems, Autonomous Navigation\\
        \textbf{Analysis Tools:} CFD, FEA, Flight Dynamics Modeling, Performance Analysis
    \end{onecolentry}

        
    \section{Experience}
        
        \begin{twocolentry}{
            June 2024 – Present
        }
            \textbf{Senior Associate Engineer}, Greenko Group -- Hyderabad, IN\end{twocolentry}

        \vspace{0.10 cm}
        \begin{onecolentry}
            \begin{highlights}
                \item Research and development of unmanned aerial systems for inspection in the renewable sector.
                \item Mission planning for drone-based inspection in wind and solar power plants.
                \item Drone data processing and automation for various software such as metashape agisoft, global mapper, DJI Terra, and Modify. along with python-based processing algorithms for custom sensor data processing.  
                \item Used open source development tool such as ROS2, NAV2, PX4, Nvidia Isaac ROS platforms, etc. for UAV automation.
                \item Development and testing of GPS denied drone for indoor inspection purpose.
                \item Research and development on sensor fusion and lidar technologies.
                \item Vendor management and selection according to project requirements.
            \end{highlights}
        \end{onecolentry}


        \vspace{0.1 cm}

        \begin{twocolentry}{
            Aug 2023 – May 2024
        }
            \textbf{Associate Engineer}, Digitele Networks (Greenko Group) -- Hyderabad, IN\end{twocolentry}

        \vspace{0.10 cm}
        \begin{onecolentry}
            \begin{highlights}
                \item Research and development of unmanned aerial systems and robotic systems.
                \item Industrial exposure on the development of robotic and UAV systems and its real-life applications.
                \item Development and testing of GPS denied drone for indoor inspection purpose.                 \item Research and development on sensor fusion and lidar technologies.
                \item Worked on thrust bench setup, VTOL development, Imaging Drone for wind turbine inspection. 
                \item Testing of various robotic and UAV systems at sites.
                \item worked with a quadruped robot for sensor integration and site inspection automation.
            \end{highlights}
        \end{onecolentry}
        
         \vspace{0.2 cm}

        \begin{twocolentry}{
            July 2022 – Aug 2023
        }
            \textbf{Graduate Engineer Trainee}, Digitele Networks (Greenko Group) -- Hyderabad, IN\end{twocolentry}

        \vspace{0.10 cm}
        \begin{onecolentry}
            \begin{highlights}
                \item Industrial exposure on the development of robotic and UAV systems and its real-life applications.
                \item Learned about the competitive development process from the mechanical as well as the software point of view.
                \item Used open source robotic development tool such as ROS, moveit, etc.
                \item Learned the use of sensors such as Velodyne LIDAR for 3d mapping of a site.
                \item Learned the use of AI-ML for automation in robotics.
            \end{highlights}
        \end{onecolentry}
        
        \vspace{0.2 cm}

        \begin{twocolentry}{
            June 2022 – Aug 2022
        }
            \textbf{Product Development Engineer (CAE), intern}, Simulation LAB -- Pune, IN\end{twocolentry}

        \vspace{0.10 cm}
        \begin{onecolentry}
            \begin{highlights}
               \item The primary focus of the internship is to develop the Battery Thermal Management System (BTMS) using nanofluids for EVs.
                \item The aim of the project is to analyze the thermal behavior of nanofluids with respect to different temperatures in our thermal network design.
                \item The project is based on the concept that maintaining temperature uniformity and reducing the maximum temperature of batteries using nanofluids. Nanofluids have high thermal conductivity and can absorb heat and cool batteries.
                \item The project is inspired by the recent events of Electric Vehicles (especially electric scooters) catching fire in India. And one of the reasons for this was the very hot weather conditions in India.
            \end{highlights}
        \end{onecolentry}

        \vspace{0.2 cm}

        \begin{twocolentry}{
            May 2019 – June 2019
        }
            \textbf{Summer Internship}, Indian Air Force (BRD-5) -- Coimbatore, IN\end{twocolentry}

        \vspace{0.10 cm}
        \begin{onecolentry}
            \begin{highlights}
                \item	Industrial exposure on the repair, maintenance, overhaul, control and operation of passenger and cargo aircraft in the IAF inventory. 
                \item	Learned about indigenization of critical components in the aircraft.
                \item	Learned about Quality Control Assurances in the IAF indigenization program.
                \item	Proposed the use of metal 3D printing to the IAF based on learning from indigenization and quality control.
                \item	Based on learning from quality control assurances, we proposed the computer vision-based dimension and tolerance test for quality check. 
                
            \end{highlights}
        \end{onecolentry}

   \section{Technologies / Skills}

        \begin{onecolentry}
            \textbf{Languages:} Python (OpenCV, NumPy, TensorFlow, Pytorch, CUDA), C++, MATLAB-simulink, Julia lang, Fortran.
        \end{onecolentry}

        \vspace{0.2 cm}

        \begin{onecolentry}
            \textbf{UAV Autopilot Technologies:} PX4, Ardupilot, ROS1/ROS2, Nvidia Isaac ROS.
        \end{onecolentry}

         \vspace{0.2 cm}

        \begin{onecolentry}
            \textbf{CAD/CAE Softwares:} Ansys workbench, Solidworks, OpenFoam, Simscale.
        \end{onecolentry}
         \vspace{0.2 cm}

        \begin{onecolentry}
            \textbf{Parametric Analysis Software:} Xflr5, Xfoil, OpenVSP, XROTOR, QPROP.
        \end{onecolentry}
        \vspace{0.2 cm}

        \begin{onecolentry}
            \textbf{Project Management tools:} Git, PMS (Similar to Jira), Notion.
        \end{onecolentry}
         \vspace{0.2 cm}

        \begin{onecolentry}
            \textbf{Operating systems:} Linux (Ubuntu, Fedora and Arch), windows, MAC.
        \end{onecolentry}
         \vspace{0.2 cm}

        \begin{onecolentry}
            \textbf{Documentation:} Microsoft Office, LaTeX.
        \end{onecolentry}
        \vspace{0.2 cm}

        \begin{onecolentry}
            \textbf{Soft Skills:} Problem solving, Team work, Adaptability, Quick learning.
        \end{onecolentry}
    
   \section{Education}

        \begin{twocolentry}{
            Sept 2020 – Aug 2022
        }
            \textbf{Moscow Aviation Institute (MAI)}, MS in Propulsion Engineering\end{twocolentry}

        \vspace{0.10 cm}
        \begin{onecolentry}
            \begin{highlights}
                \item GPA: 4.0/4.0
            \end{highlights}
        \end{onecolentry}
        
        \vspace{0.20 cm}
        
        \begin{twocolentry}{
            Jul 2016 – Jun 2020
        }
            \textbf{University of Petroleum and Energy Studies (UPES)}, B.tech in Aerospace Engineering with spz. in Avionics\end{twocolentry}

        \vspace{0.10 cm}
        \begin{onecolentry}
            \begin{highlights}
                \item CGPA: 7.4/10
            \end{highlights}
        \end{onecolentry}


    \section{Projects}

        \vspace{0.10 cm}
        \begin{twocolentry}{
            2022
        }
            \textbf{Master's Thesis (Numerical Analysis of film cooling in high Pressure turbine blade)}\end{twocolentry}

        \vspace{0.10 cm}
        \begin{onecolentry}
            \begin{highlights}
                \item The research aims to investigate the cooling requirements in high-pressure turbine blades in gas turbine engines. Lastly, a reduction of heat transfer into the blade by using film cooling over the blade.
                \item The analysis was done for 2D flat plate and then for 3D turbine blade geometry.
                \item The blade was modeled using solid works and the simulation was conducted using ANSYS Fluent software.
                \item The Film was found to be 27 percent effective near leading edge and 47 percent effective near trailing edge.
                \item Improved public speaking and presentation skills during thesis defense.
            \end{highlights}
        \end{onecolentry}


        \vspace{0.2 cm}

        \begin{twocolentry}{
            2022
        }
            \textbf{Development of Numerical solvers using python Programming- Coursework}\end{twocolentry}

        \vspace{0.10 cm}
        \begin{onecolentry}
            \begin{highlights}
                \item This coursework aims to develop Python-based solvers for various numerical analysis problems.
                \item The core python programming with Numpy, Sympy, Scipy, Matplotlib, Object-oriented programming, etc. was used.
                \item Jupyter notebooks and anaconda were used throughout the coursework.
                \item Developed a thorough understanding of the Python programming language.
            \end{highlights}
        \end{onecolentry}


        \vspace{0.2 cm}

        \begin{twocolentry}{
            2021
        }
            \textbf{Design and simulation of gas turbine engine combustion chamber - Coursework}\end{twocolentry}

        \vspace{0.10 cm}
        \begin{onecolentry}
            \begin{highlights}
                \item This project work aims to design the GTE combustion chamber based on source data of Russian gas turbine engine and to understand the combustion of Jet-A fuel and ambient air.
                \item Hand calculation of chamber inlet (compressor exit) and chamber outlet (Turbine inlet) conditions was performed to use in simulation software (ANSYS CFX).
                \item Based on the source data and hand calculation the geometry was created on paper and then modeled using solid works and simulated using ANSYS CFX.
                \item By varying number of air vents in inner combustion chamber shell the desired flow behavior and combustion was achieved.
                \item Developed thorough understanding of ANSYS CFX simulation tool.
            \end{highlights}
        \end{onecolentry}

        \vspace{0.1 cm}

        \begin{twocolentry}{
            2020
        }
            \textbf{Study and Numerical Analysis of Pre-Mixed Methane Flame Stabilization for Micro-Gas turbine Engine (Major Project)}\end{twocolentry}

        \vspace{0.1 cm}
        \begin{onecolentry}
            \begin{highlights}
                \item This project aims to simulate the Pre-Mixed Methane-Air flame using ANSYS Fluent. And stabilization of that flame by introducing the Bluff body (or blunt-body) inside the combustion chamber of a Micro Gas-Turbine engine.
                \item For the simulation Ansys fluent and CHEMKIN tool was used.
                \item By the use of circular, semicircular and crescent body the stabilization of flame was achieved by creating recirculation zone.
                \item Developed a thorough understanding of ANSYS Fluent Simulation tool and CHEMKIN chemical kinetics equations solver.
            \end{highlights}
        \end{onecolentry}

        \vspace{0.1 cm}

        \begin{twocolentry}{
            2019
        }
            \textbf{Customizing in-House Production Unit with Additive Manufacturing Process (Summer Internship Project)}\end{twocolentry}

        \vspace{0.10 cm}
        \begin{onecolentry}
            \begin{highlights}
                \item This project aims the replacement of the manufacturing process used in IAF with additive Manufacturing (3D Printing) process. Which can help reduce the manufacturing time and cost of Indigenized parts at IAF.
                \item As large number of aircrafts in IAF inventory are of old Russian origin. The MRO of critical components is challenging. The metal 3D printing of aerospace materials was proposed for components such as Screws, joints, frames, handles, etc.
                \item Developed understanding of challenges faced by IAF in foreign origin systems.
            \end{highlights}
        \end{onecolentry}

        \vspace{0.2 cm}

        \begin{twocolentry}{
            2017-2019
        }
            \textbf{Design and development of flapping-wing Nano-sized aerial vehicle (Nano-Fly)}\end{twocolentry}

        \vspace{0.10 cm}
        \begin{onecolentry}
            \begin{highlights}
                \item This project aims to design and develop a nature-inspired bio-mimetic ornithopter with the capability of performing surveillance operations.
                \item Performed the mechanical hand calculations for gear box, used to convert rotational motion to flapping motion. The gear assembly was design to achieve approximately 50-70 flaps per second.
                \item Crank and rocker mechanism were used to convert motor rotation to wing flaps.
                \item A short-range remote-controlled flight was achieved by Nano-fly.
                \item Developed understanding of micro actuation systems (MEMS; Micro-electromechanical systems) such as piezo-electric systems.
                \item Developed critical design thinking and team work skills in this project.
            \end{highlights}
        \end{onecolentry}
        
        \vspace{0.2 cm}

        \begin{twocolentry}{
            2017-2018
        }
            \textbf{Team Agastya - Development of flying wing design transport aircraft}\end{twocolentry}

        \vspace{0.10 cm}
        \begin{onecolentry}
            \begin{highlights}
                \item	Represented UPES, Dehradun in the AIAA DBF (American Institute of Aeronautics and Astronautics: Design build and Fly) competition for the first time.
                \item	Responsible for managing Aircraft design and selecting propulsion system for the Aircraft.
                \item	Proposed various design changes such as use of flying wing design for better stability.
                \item	Evaluated various options for manufacturing of the aircraft and used technologies such as 3D printing, water jet cutter and hot wire cutter.
                \item	Compiled a list of parts suppliers and developed a knowledge sharing system for effective continuation of the project by subsequent teams.
                
            \end{highlights}
        \end{onecolentry}
    
   % \section{Declaration}

        %\begin{onecolentry}
        %    I hereby declare that all the information given above is true and correct to the best of my knowledge.
        %\end{onecolentry}

\end{document}